% !TeX spellcheck = ru_RU
% !TEX root = vkr.tex

\label{sec:relatedworks}
Так как изначально было понятно, в каком виде должен был выглядеть сайт (такой же интерфейс, как в Wireshark),
задачей стало именно нахождение аналога Wireshark, но в виде веб-приложения, для того, чтобы его можно было внедрить 
в Miminet 
% ~\cite{parallel}. 

\subsection{CloudShark}
Одним из наилучших аналогов оказался CloudShark~\cite{CloudShark}.
Очень удобный и понятный проект, в котором реализованы все функции Wireshark.
Единственный и существенный минус этого сервиса -- он платный.
В дальнейшем именно этот сайт стал референсом для реализации продукта. 

\subsection{PacketSafari}
PacketSafari~\cite{PacketSafari} - также является хорошим веб-анализатором трафика, но также является платным.

\subsection{PacketTotal}
PacketTotal~\cite{PacketTotal} - бесплатный сервис с открытым api, но с неудобным интерфейсом, который не подходит для проекта.

У всех представленных сервисов очень много лишнего функционала, который не является необходимым в конечном продукте.

Исходя из собранной информации, было принято решение реализовать собственный сервис отображения пакетов.