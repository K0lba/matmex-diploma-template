\subsection{Flask}
Для реализации сервиса был выбран Flask~\cite{Flask}.
Flask -- фреймворк для создания веб-приложений на языке программирования Python, использующий набор инструментов Werkzeug, а также шаблонизатор Jinja2.
Это гибкий и легкий инструмент, который позволяет облегчить процесс создания веб-приложений на Python.
Flask -- легкий, в отличии от его аналога Django. Он предоставляет только самые базовые инструменты,
что отлично подходит для реализации небольшого веб-сервиса. 
Miminet также написан с использованием Flask.


\subsection{Bootstrap}
Для написания веб-приложения был выбран фреймворк Bootstrap5~\cite{Bootstrap}.
Bootstrap - свободный набор инструментов для создания сайтов 
и веб-приложений. Включает в себя HTML- и CSS-шаблоны оформления для типографики, 
веб-форм, кнопок, меток, блоков навигации и прочих компонентов веб-интерфейса, 
включая JavaScript-расширения.
Он позволяет создать красивый дизайн приложения за короткое время. 
Еще одной причиной выбора именно этого фреймворка стало то, что Miminet разработан именно с использованием Bootstrap.


\subsection{dpkt} 
Для анализа PCAP~\cite{PCAP} файлов была выбрана очень простая в освоении библиотека для Python -- dpkt~\cite{dpkt}.
dpkt — это модуль Python для быстрого и простого 
создания/анализа пакетов с определениями для основных 
протоколов TCP/IP.
dpkt также, как и Flask и Bootstrap, использовался при написании Miminet.
