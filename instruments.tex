\subsection{OCaml} OCaml --- объектно-ориентированный язык
функционального программирования общего назначения.
Был разработан с учётом безопасности исполнения и надёжности 
программ. Поддерживает функциональную, императивную
и объектно-ориентированную парадигмы программирования. 

Многоядерный OCaml --- это расширение OCaml со
встроенной поддержкой параллелизма, с
общей памятью (SMP), с помощью доменов и
параллелизма с помощью алгебраических эффектов. 
OCaml 5.0 первый 
выпуск, официально поддерживающий многоядерность.

Для реализации проекта выбрана самая свежая версия OCaml (5.0.0 beta2)~\cite{ocaml}.
В этой версии добавили multicore в runtime, c помощью которого можно
писать параллельно исполняющиеся программы на самом OCaml. В отличии от 4 версии
Ocaml, где приходится запускать системные потоки и параллельность происходит на уровне системы, 
а не на уровне OCaml.

\subsection{Domainslib} Для распараллеливания в OCaml 5.0 
используется библиотека Domainslib, в 
% ссылка
которой присутствуют 2 модуля: Task --- для вызова
многопоточности, и Chan --- для передачи информации между
доменами. Это гибкий и легкий инструмент, который позволяет 
облегчить процесс реализации параллелизма.

\subsection{miniKanren} miniKanren --- это семейство 
языков программирования для реляционного программирования. 
Поскольку отношения являются двунаправленными, если miniKanren 
задано выражение и желаемый результат, miniKanren может выполнить 
выражение «назад», найдя все возможные входные данные для 
выражения, которые дают желаемый результат.
Вся проделанная работа выполнена на минимальной реализации
miniKanren --- Unicanren, состоящей из 4 базовых конструкций:
fresh --- объявления новой переменной, unify --- присваивания, cond$^e$ --- дизъюнкции и conj --- конъюнкции.
% ccылка на миниканрен,Ю кандидатскую бернда
