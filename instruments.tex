3.1 OCaml — объектно-ориентированный язык
функционального программирования общего назначения.
Был разработан с учётом безопасности исполнения и надёжности 
программ. Поддерживает функциональную, императивную
и объектно-ориентированную парадигмы программирования. 
Самый распространённый в практической работе диалект языка ML. 

Многоядерный OCaml — это расширение OCaml со
встроенной поддержкой параллелизма с
общей памятью (SMP) с помощью доменов и
параллелизма с помощью алгебраических эффектов. 
Он сливается с транком OCaml. OCaml 5.0 первый 
выпуск, официально поддерживающим многоядерность.

Для реализации проекта выбрана самая свежая версия OCaml (5.0.0 beta2)1
В этой версии добавили multicore в runtime, c помощью которого можно
писать параллельно исполняющиеся программы на самом OCaml, в отличии от 4 версии
Ocaml, где приходится запускать системные потоки и параллельность происходит на уровне системы, 
а не на уровне OCaml

3.2 Для распараллеливания используется библиотека DomainsLib, в
которой присутствуют 2 модуля: Task – для вызова
многопоточности и Chan – для передачи информации между
доменами.Это гибкий и легкий инструмент, который позволяет 
облегчить процесс реализации параллелизма, в отличии отсвоих 
аналогов Eio, Reagents, Kcas, в которых тяжело разобораться

3.3 miniKanren — это семейство 
языков программирования для реляционного программирования. 
Поскольку отношения являются двунаправленными, если miniKanren 
задано выражение и желаемый результат, miniKanren может выполнить 
выражение «назад», найдя все возможные входные данные для 
выражения, которые дают желаемый результат.
Вся проделанная работа выполнена на минимальной реализации
miniKanren – Unicanren, состоящую из 4 базовых конструкций:
Fresh --- объявления новой переменной, Unify --- присваивания, Cond$^e$ --- дизъюнкции и Conj --- конъюнкции.
