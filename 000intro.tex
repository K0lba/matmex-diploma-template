% !TeX spellcheck = ru_RU
% !TEX root = vkr.tex


Реляционное программирование – это вид декларативного программирования, в рамках которого программы представляются как набор
отношений. Отношения не делают различий между входными и выходными параметрами. Благодаря этому одна и та же реляционная
программа может использоваться для решения нескольких связанных
проблем. Одним из представителей реляционной парадигмы является
язык программирования MiniKanren. MiniKanren состоит из небольшого числа примитивов. Благодаря этому он может быть легко реализован как встраиваемый предметно-ориентированный язык[5]. Одной из
таких реализаций является unicanren1
, встроенный в функциональный
язык OCaml.

Добавление параллелизации -- это важная новость в мире OCaml, острие науки об этом пишут статьи,
поэтому тема является неизученной и актуальной для дальнейщих улучшений miniKanren

Параллелизм — это то, как мы разделяем несколько вычислений таким образом, чтобы они могли выполняться в перекрывающиеся периоды времени, а не строго последовательно. Параллелизм — это одновременное выполнение нескольких вычислений, в первую очередь за счет использования нескольких ядер на многоядерной машине.  

Распараллеливание на многоядерные процессоры позволяет ускорять вычислительно нагруженные программы, конкурентность облегчает написание программ с активно взаимодействующими между собой и с другими программами потоками

Постоянно предпринимаются усилия по созданию программных абстракций, облегчающих использование многоядерного оборудования. Многие программные абстракции (например, параллельные объекты, транзакционная память и т. д.) упрощают дело, но по-прежнему требуют сложной инженерии. Утверждается, что некоторые трудности многоядерного программирования можно облегчить с помощью декларативного стиля программирования, в котором программисты напрямую выражают независимость фрагментов последовательных программ.

В рамках данной работы предлагается разработать решение,параллелизирующее miniKanren с целью повышения эффективности выполонения программ
\blfootnote{
	Дата сборки: \today\\}
