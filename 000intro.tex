% !TeX spellcheck = ru_RU
% !TEX root = vkr.tex


Реляционное программирование --- это вид декларативного программирования, в рамках которого программы представляются как набор
отношений. Отношения не делают различий между входными и выходными параметрами. Благодаря этому одна и та же реляционная
программа может использоваться для решения нескольких связанных
задач. Одним из представителей реляционной парадигмы является
язык программирования mniKanren. miniKanren состоит из небольшого числа примитивов. Благодаря этому он может быть легко реализован как встраиваемый предметно-ориентированный язык~\cite{moiseenko_podkopaev}. 


% Параллелизм --- это разделение вычислений на независимые части, которые исполняются не последовательно, а параллельно. 
Параллелизм --- это одновременное выполнение нескольких вычислений, в первую очередь за счет использования нескольких ядер на многоядерной машине.  
Процесс --- это исполняющаяся программа.Процесс имеет связанные с ним системные ресурсы --- это его память, открытые им файлы, открытые сетевые соединения и прочие подобные системные ресурсы.
Поток --- это часть процесса, соответствующая потоку исполнения.
Многопоточные программы состоят из нескольких потоков в рамках одного процесса. 
У них процесс общения между потоками очень быстр 
(хотя и медленнее, чем обращение к памяти из последовательной программы), 
передача данных занимает наносекунды~\cite{parallel}. 

Конкурентность ---  устанавливает 
отдельные точки исполнения вычислений или процессов, 
называемые управляющими потоками. Они позволяют этим вычислениям избежать ожидания завершения всех 
остальных вычислений --- как это происходит в случае последовательного программирования.

Хотя и считается, что конкурентные вычисления включают в себя параллельные, у них есть существенные отличия.

Параллельные вычисления используют более одного вычислительного ядра, поскольку все управляющие потоки работают одновременно и 
занимают весь рабочий цикл ядра на время исполнения — именно поэтому параллельное вычисление невозможно на одноядерном компьютере. 
В этом они отличаются от конкурентных вычислений, которые фокусируются на пересечениях жизненных циклов вычислений. 
Например, этапы выполнения процесса могут быть разбиты на временные промежутки, и если процесс не заканчивает своё существование 
до конца промежутка, он приостанавливается, предоставляя другому процессу возможность работать.

Главным преимуществом этого подхода является максимально возможное использование ресурсов системы~\cite{concurrency}.

Распараллеливание на многоядерные процессоры позволяет ускорять вычислительно нагруженные программы, 
конкурентность облегчает написание программ с активно взаимодействующими между собой и с другими программами, потоками.

Постоянно предпринимаются усилия по созданию программных
абстракций, облегчающих использование многоядерного оборудования.
Многие программные абстракции (например, параллельные объекты, 
транзакционная память и т. д.) упрощают дело, но по-прежнему 
требуют сложной инженерии. Утверждается, что некоторые трудности 
многоядерного программирования можно облегчить с помощью 
декларативного стиля программирования, в котором 
программисты напрямую выражают независимость фрагментов 
последовательных программ.

В рамках данной работы предлагается разработать решение,
параллелизирующее miniKanren с целью повышения 
эффективности выполонения программ.

\blfootnote{
	Дата сборки: \today\\}
