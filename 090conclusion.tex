% !TeX spellcheck = ru_RU
% !TEX root = vkr.tex

% \textbf{Обязательно для промежуточного, полугодового, годового и  любых других отчётов.}

В результате работы была реализована параллелизация miniKanren на языке OCaml
Были выполнены поставленные задачи:

% \textbf{Для практик/ВКР.} Также важно сделать список результатов, который будет один к одному соответствовать задачам из раздела~\ref{sec:task}.

\begin{itemize}
\item Выбрана версия Ocaml и библиотека для параллелизации
\item Реализована втроенная параллелизация Unicanren(miniKanren)
\item Проведены тесты и сделан вывод об эффективности реализации
\end{itemize}


\textbf{Дальнейшие планы}

Также был составлен план дальнейших действий и выделены части,
требующие доработки, а именно тестирование на более крупных входных данных, а также рефакторинг реализации

% \noindent Для промежуточных отчетов сюда важно записать какие задачи уже были сделаны за осенний семестр, а какие только планируется сделать.

% Также сюда можно писать планы развития работы в будущем, или, если их много, выделять под это отдельную главу.

