% !TeX spellcheck = ru_RU
% !TEX root = vkr.tex

% \textbf{Обязательно для промежуточного, полугодового, годового и  любых других отчётов.}

В результате работы был реализован анализатор сетевых пакетов для Miminet.
Были выполнены поставленные задачи:

% \textbf{Для практик/ВКР.} Также важно сделать список результатов, который будет один к одному соответствовать задачам из раздела~\ref{sec:task}.

\begin{itemize}
    \item  проведен обзор веб-аналогов Wireshark;
    \item  спроектирован сервис;
    \item  написана веб-страница;
    \item  информация о пакетах выведена на страницу;
    \item  конечный продукт интегрирован в Miminet.
    % \item Проведены тесты и сделан вывод об эффективности реализации.
\end{itemize}


Дальнейшие планы:
\begin{itemize}
    \item  реализовать некоторые функции (фильтры поиска и т.д.);
    \item  добавить возможность импорта/экспорта данных.
\end{itemize}



Всю проделанную работу можно увидеть на GitHub репозитории~\footnote{\url{https://github.com/K0lba/MimiShark} (Дата доступа: 2023-05-09)}.

% \subsection{ \textbf{Дальнейшие планы}}

% Также был составлен план дальнейших действий и выделены части,
% требующие доработки, а именно тестирование на более крупных входных данных, а также рефакторинг реализации

% \noindent Для промежуточных отчетов сюда важно записать какие задачи уже были сделаны за осенний семестр, а какие только планируется сделать.

% Также сюда можно писать планы развития работы в будущем, или, если их много, выделять под это отдельную главу.

